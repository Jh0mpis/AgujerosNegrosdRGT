\chapter{Gravedad masiva de Rahm,  Gabadadze y Tolley}
\label{chap:dRGT}

El dotar de masa al gravitón fue una propuesta bastante llamativa debido a que da una posible interpretación al problema de la constante cosmológica \cite{CosmologicalStudy1}. Sin embargo, si bien el construir una teoría de partículas masivas de espín-2 no es una tarea imposible \cite{MassiveGravity,TheoreticalAspectsOfMassiveGRavity}, cuando se realizan interacciones entre las partículas estas acarrean varías complicaciones, como la discontinuidad vDVZ y los fantasmas de Boulware-Deser \cite{MassiveGravity}. La propuesta de gravedad masiva de de Rham, Gabadadze y Tolley (dRGT) es una propuesta de gravedad masiva no lineal que resuelve la dicontinuidad y  los campos fantasmas en la teoría de gravedad masiva de Fierz Pauli.\\

A lo largo del capitulo se abordará la gravedad masiva dRGT, la acción modificada y las ecuaciones  de campo resultante de esta.

\section{Gravedad masiva dRGT}

Inicialmente, una teoría que busque ser una alternativa a la relatividad general debe cumplir con algunas propiedades básicas, una de ellas es que se conserve el difeomorfismo en la descripción de la teoría, si bien el término de Fierz-Pauli de la ecuación \eqref{eq:Fierz-Pauli} no conserva este difeomorfismo de la teoría; se introducen los campos de St\"{u}ckelberg (ecuación \eqref{eq:Fierz-Pauli-Stuckelberg}) para recuperar este invariante \cite{CosmologicalStudy1}.\\

Además, para el caso de las teorías de gravedad masiva estas deben recuperar la relatividad general en el caso de $m_g\rightarrow 0$, cosa que como se vio en el capítulo anterior no sucede con el término de Fierz-Pauli, ni siquiera usando el truco de St\"{u}ckelberg \cite{MassiveGravity}. Sin embargo, una solución de este problema fue conjeturada por Vainshtein, quién argumentó que la relatividad general puede ser recuperada a distancias pequeñas incluyendo términos no lineales en la teoría de campo. Esto debido a que los terminos de orden superior terminan apantallando a los de orden lineal \cite{VainshteinSol}.\\

Finalmente, como se comentó en el capítulo anterior, una teoría de gravedad masiva consistente físicamente debe estar libre de fantasmas Boulware-Deser, en una teoría no lineal de gravedad masiva se tienen 5 grados de libertad (asociados con las helicidades de la partícula) y un grado de libertad adicional (Boulware-Deser), el cual genera términos de energía negativa que hacen la teoría inestable \cite{CosmologicalStudy1}. La solución para este problema con los grados de libertad es ajustar la teoría de tal manera que todas las derivadas de los modos de helicidad-0 sean derivadas totales, esta imposición elimina el grado de libertad adicional reduciendo el orden de las ecuaciones diferenciales \cite{MassiveGravity}.\\

Por tanto, la gravedad masiva dRGT, es una teoría de gravedad no lineal que no presenta la discontinuidad vDVZ y además es una teoría libre de fantasmas Boulware-Deser. Esta a su vez se puede extender a una teoría de ``bi-gravedad'', lo cual implica que se tienen dos métricas $g_{\mu\nu}$ y $f_{\mu\nu}$ (asociada a los campos  de St\"{u}ckelberg) que interactúan entre ellas \cite{GhostFreeMassiveGravity,BlackHolesInMG}.\\

\section{Acción de Einstein-Hilbert modificada}

Una acción para una teoría de gravedad masiva covariante genérica es obtenida sumando un potencial a la acción de Einstein-Hilbert \cite{GhostFreeMassiveGravity},

\begin{equation}
    S=\dfrac{M^2_{Pl}}{2}\int \sqrt{-g}(\mathcal{L}_{EH}+\mathcal{L}_{MG})d^4x=\dfrac{M^2_{Pl}}{2}\int \sqrt{-g}\left(R-\dfrac{m_g^2}{2}\mathcal{U}[g^{-1}f]\right) d^4x,
    \label{eq:Accion1}
\end{equation}

donde, $M_{Pl}$ es la escala de Plank, $R$ el escalar de Ricci en el término de la acción de Einstein-Hilbert, $m_g$ la masa del gravitón y $\mathcal{U}[g^{-1}f]$ un funcional en término de las dos métricas, la métrica $f_{\mu\nu}$ se denomina comúnmente de referencia, esta puede escogerse de diferentes maneras (más información en \cite{GhostFreeMassiveGravity}).\\

Se define el tensor $H_{\mu\nu}$ como la covariantización de la perturbación sobre la métrica $g_{\mu\nu}=\eta_{\mu\nu}+h_{\mu\nu}=H_{\mu\nu}+f_{ab}\partial_\mu\phi^a\partial_\nu\phi^b$, donde $\phi^a$ son los cuatro campos de St\"{u}ckelberg \cite{ResummationOfMG,GeneralizationOfFP}.\\

Ahora, se define el tensor

\begin{equation}
    \mathcal{K}^\mu_\nu=\delta^\mu_\nu-\sqrt{\delta^\mu_\nu-H^\mu_\nu}=\delta^\mu_\nu-\sqrt{g^{\mu\sigma}f_{ab}\partial_\sigma\phi^a\partial_{\nu}\phi^{b}},\label{eq:Kdefinition}
\end{equation}

de tal manera que $\left.\mathcal{K}_{\mu\nu}\right|_{h_{\mu\nu}=0}=\Pi_{\mu\nu}$, con $\Pi_{\mu\nu}=\partial_\mu\partial\nu\pi$ el campo dentro del término de  masa de Fierz-Pauli y $\pi$ el modo de helicidad-0 de la partícula en el campo \cite{ResummationOfMG,GeneralizationOfFP}.\\

Como una extensión del término de Fierz-Pauli, se puede escribir una función general en términos de este, esto es
\begin{equation}
    \mathcal{L}_{MG}=\dfrac{m_g^2}{2}\mathcal{U}[\Pi_{\mu\nu}^2-\Pi^2],
\end{equation}
lo anterior usando la ecuación \eqref{eq:Accion1}. Usando este término es posible eliminar los fantasmas de Boulware-Deser ajustando que ($\Pi_{\mu\mu})^n$ son derivadas totales \cite{MassiveGravity}. Dado que $\mathcal{K}_{\mu\nu}$ es una perturbación de $\Pi_{\mu\nu}$, $\mathcal{U}$ se puede escribir como $\mathcal{U}[\mathcal{K}_{\mu\nu}^2-\mathcal{K}^2]$. Sin embargo, cada $\mathcal{K}_{\mu\nu}^2-\mathcal{K}^2$ hace parte de un lagrangiano $\mathcal{L}^{(n)}$ con $n$ el orden de la perturbación, estos se pueden escribir como \cite{GeneralizationOfFP}
\begin{equation}
    \mathcal{L}^{(n)}=-\sum\limits_{m=1}^n(-1)^m\dfrac{(n-1)!}{(n-m)!}[\mathcal{K}^m]\mathcal{L}^{(n-m)},
\end{equation}
donde $[\mathcal{K}]=\mathcal{K}^{\mu}_\nu$ la traza de $\mathcal{K}$. Luego podemos escribir el lagrangiano total como \cite{GeneralizationOfFP,ResummationOfMG}
\begin{equation}
    \dfrac{m_g^2}{2}\mathcal{U}[\mathcal{K}]=-\dfrac{m_g^2}{2}\sum\limits_{n=2}^4\alpha_n \mathcal{L}^{(n)}(\mathcal{K})=-\dfrac{m_g^2}{2}\sum\limits_{n=2}^4\alpha_n \mathcal{U}_n(\mathcal{K}).
    \label{eq:Udefinition}
\end{equation}

A continuación, con $\mathcal{L}^{(0)}=1$ y $\mathcal{L}^{(1)}=[\mathcal{K}]$, con $\mathcal{K}$ definido por la ecuación \eqref{eq:Kdefinition} \cite{ResummationOfMG}, se obtienen

\begin{gather}
    \label{eq:U2}
        \mathcal{U}_2=[\mathcal{K}]^2-[\mathcal{K}^2],\\
        \mathcal{U}_3=[\mathcal{K}]^3-3[\mathcal{K}][\mathcal{K}^2]+2[\mathcal{K}^3] \label{eq:U3},
        \\
        \mathcal{U}_4=[\mathcal{K}]^4-6[\mathcal{K}]^2[\mathcal{K}^2]+8[\mathcal{K}][\mathcal{K}^3]+3[\mathcal{K}^2]^2-6[\mathcal{K}^4],
\end{gather}

para los ordenes de $n\geq  5$ se tiene  $\mathcal{U}_{n\geq 5}(\mathcal{K})=0$, pues introducir factores de orden superior en $\mathcal{U}$ reintroduce fantasmas en la teoría \cite{StabilityOfSdSBlackHoles}.\\

Finalmente la acción para la teoría de bi-gravedad masiva dRGT, ajustando $\dfrac{\alpha_2}{2}=1$ en \eqref{eq:Udefinition}, queda escrito de la forma

\begin{equation}
    S_{dRGT}=\dfrac{M^2_{Pl}}{2}\int \sqrt{g}\left[R+m_g^2\left(\mathcal{U}_2+\alpha_3\mathcal{U}_3+\alpha_4\mathcal{U}_4\right)\right] d^4x.
    \label{eq:AcciondRGT}
\end{equation}

Una de las características más importantes de la teoría de gravedad masiva dRGT es su conexión natural con la constante cosmológica a través de la masa del gravitón \cite{TheoreticalAspectsOfMassiveGRavity}.

\section{Ecuaciones de campo modificadas en la teoría dRGT}

Tras obtener la forma de la nueva acción, el paso siguiente sería obtener las nuevas ecuaciones de campo. Para obtener estas es necesario minimizar la acción variando respecto de la métrica $g_{\mu\nu}$, es decir

\begin{equation}
    \delta S=\delta S_{EH}+\delta S_{m_g}=0.
    \label{eq:VarAction}
\end{equation}

Dado que el primer término de la ecuación \eqref{eq:VarAction} corresponde a la variación de la acción de Einstein-Hilbert, este corresponderá al tensor de Einstein $G_{\mu\nu}$ en las ecuaciones de campo, mientras que para el segundo término se tendrá un nuevo tensor que se denotará como $X_{\mu\nu}$ \cite{ConsistentMassiveGraviton}. Por tanto, las ecuaciones de campo se pueden escribir como 

\begin{equation}
    \label{eq:fieldEquations}
    G_{\mu\nu}+m_g^2X_{\mu\nu}=0,
\end{equation}

donde, el tensor $X_{\mu\nu}$ puede ser interpretado como un tensor de momento-energía efectivo de las partículas de espín 2.\\

El tensor $X_{\mu\nu}$ resulta de la variación del término $\delta S_{m_g}$, por tanto, este termino puede escribirse como \cite{BlackHoleSolutionIndeRGTMassiveGravity,StabilityOfSdSBlackHoles},

\begin{equation}
    X_{\mu\nu}=\sqrt{-g}\dfrac{\delta(\sqrt{-g}\mathcal{U})}{\delta g^{\mu\nu}}=\dfrac{\delta\mathcal{U}}{\delta g^{\mu\nu}}-\frac{1}{2}\mathcal{U}g_{\mu\nu}\label{XVar},
\end{equation}

donde $\mathcal{U}$ viene dado según la ecuación \eqref{eq:Udefinition}. Realizando la variación de $\mathcal{U}$ respecto de la métrica $g_{\mu\nu}$ y escribiendo la expresión en términos de las trazas de $\mathcal{K}^n$, el tensor $X_{\mu\nu}$ queda escrito de la forma  \cite{NoHairTheoremInQuasi},

\begin{equation}
    \begin{split}
        X_{\mu\nu}=\mathcal{K}_{\mu\nu}-[\mathcal{K}]g_{\mu\nu}-\alpha\left\{\mathcal{K}^2_{\mu\nu}-[\mathcal{K}]\mathcal{K}_{\mu\nu}+\dfrac{\mathcal{U}_2}{2}g_{\mu\nu}\right\}\\
        +3\beta\left\{\mathcal{K}^3_{\mu\nu}-[\mathcal{K}]\mathcal{K}^2_{\mu\nu}+\dfrac{\mathcal{U}_2}{2}\mathcal{K}_{\mu\nu}-\dfrac{\mathcal{U}_3}{6}g_{\mu\nu}\right\},
    \end{split}
\label{eq:TensorMomentoEnergíaEfectivo}
\end{equation}

donde $\mathcal{U}_2$ y $\mathcal{U}_3$ están definidos en las ecuaciones \eqref{eq:U2} y \eqref{eq:U3} respectivamente. Además, se introdujeron dos parámetros nuevos $\alpha$ y $\beta$ como sigue \cite{AClassOfBlackHoles},

\begin{equation}
    \alpha_3=\dfrac{\alpha-1}{3}, \qquad \alpha_4=\dfrac{\beta}{4}+\dfrac{1-\alpha}{12}.
\end{equation}

Además de las ecuaciones de campo, es posible imponer las identidades de Biachi, esto debido a que $\nabla^\mu G_{\mu\nu}=0$, por tanto, aplicando la derivada covariante en la ecuación \eqref{eq:fieldEquations}, se impone sobre el campo \cite{AClassOfBlackHoles, BardeenBlackHole}

\begin{equation}
    \nabla^{\mu}X_{\mu\nu}=0.
\end{equation}